% Complete recipe example
\begin{recipe}
[% 
    preparationtime = {\unit[10]{min} + \unit[5]{min} per Person},
    portion = {\portion{2-4}}
]
{Okonomiyaki}
    
    \graph
    {% pictures
        small=pic/okonomiyakicabbage,     % small picture
        big=pic/okonomiyaki  % big picture
    }
    
    \introduction{%
        A Japanese cabbage pancake with a tasty sauce. I have no idea about the ratio of the ingredients. This one is with corn and cheese, but there are many possibilities of what you can add.
    }
    
    \ingredients{%
        500 g cabbage \\
        3 eggs\\
        onion \\
        flour \\
        water \\
        oil for frying\\      
        a can of corn (optional)\\
        gouda (optional)\\
        mayonaise \\
        Okonomiyaki sauce \\
    }
    
    \preparation{%
        \step Mix egg, some salt and pepper, flour and water to make a pancake dough, that is kinda liquid.
        \step Cut the cabbage and the onion into small pieces and add them to the pancake dough
        \step Put some amount of the mix into a pan, to make a pancake, that is as thin as possible without falling apart. . While the upper side is still liquid, put some corn and press it into the dough, so that the corn sinks in.
        \step When the upper side of the pancake seems also done, flip the pancake. If this side is done, flip it back and immediately put some cheese.
        \step Add the sauces and keep warm if you make more.  
    }
    
    \suggestion[]
    {%
        
    }
    
    
    \hint{%
         This recipy will demand inviting friends to get rid of the cabbage. At least it lasts quite a while and it's surprisingly quick to make just one.  \#fast \#for1 
    }
    
\end{recipe}